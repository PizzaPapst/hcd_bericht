\chapter{Einleitung}

Wer sich draußen bewegt, ob in der Natur oder in der Stadt, erlebt Freiheit, Bewegung und Ausgleich zum Alltag. Dabei spielt Sicherheit eine zentrale Rolle. Im Rahmen des Moduls „Human-Centered Design“ befasst sich die vorliegende Projektarbeit mit der Entwicklung einer menschenzentrierten Sicherheits-App für Outdoor-Aktivitäten. Sie soll speziell auf die Bedürfnisse von Personen bei Aktivitäten im Freien wie Laufen, Radfahren oder Spazierengehen zugeschnitten sein. Dabei steht nicht nur der Schutz der aktiven Personen im Fokus, sondern auch die Berücksichtigung ihrer Angehörigen (Familie/Freundeskreis), um ein durchgängiges Sicherheitsgefühl während der Outdoor-Aktivitäten zu gewährleisten.

Die Projektarbeit gliedert sich neben der Einleitung in sechs Kapitel. Den Einstieg bildet die Analysephase, in der mittels Stakeholderanalyse und einer Online-Umfrage die spezifischen Nutzeranforderungen und -wünsche für die App-Entwicklung identifiziert werden. Aufbauend auf diesen Daten erfolgt in \autoref{app:konzeption} mit der Konzeption die Überführung in Personas und Nutzungsszenarien sowie eine Aufgabenanalyse-Matrix. Die darauffolgende Entwicklungsphase widmet sich der Gestaltung des Interface-Designs sowie der Erstellung eines interaktiven Prototyps. Im Anschluss wird dieser Prototyp einem Nutzertest unterzogen, um daraus fundierte Optimierungsmaßnahmen abzuleiten. Zum Schluss folgt eine Betrachtung des Einführungsprozesses sowie ein Fazit und Ausblick.



