\chapter{Fazit und Ausblick}
Ziel dieser Projektarbeit war die menschenzentrierte Entwicklung einer Sicherheits-App für Outdoor-Aktivitäten. Das Konzept sollte anschließend in Form eines interaktiven Dummys prototypisch umgesetzt und durch einen Benutzertest evaluiert werden. Zu diesem Zweck wurde zunächst eine Stakeholderanalyse durchgeführt. Sie verdeutlichte, dass sich die Anwendung an sehr unterschiedliche Nutzergruppen richtet, die trotz variierender Aktivitäten ein gemeinsames Grundbedürfnis nach Sicherheit im Outdoor-Bereich teilen. Gleichzeitig unterscheiden sich die Anforderungen hinsichtlich Funktionen, technischer Zuverlässigkeit, Bedienbarkeit und Datenschutz deutlich. Die anschließend durchgeführte Online-Umfrage zeigte mit einer Nutzungsbereitschaft von 78 \% einen klaren Bedarf an einer Sicherheits-App für Outdoor-Aktivitäten, insbesondere bei jüngeren, überwiegend weiblichen Nutzenden. Frauen fühlen sich häufiger unsicher und priorisieren Funktionen wie Notfall-SOS, Live-Tracking und akustische Alarme. Männer bevorzugen hingegen eher Autonomie- und Informationsfunktionen. Die App sollte daher als intuitiver, zuverlässiger und datenschutzkonformer Begleiter mit starker Offline-Funktionalität gestaltet werden.

Auf Basis der Ergebnisse der Analysephase wurde das App-Konzept anschließend prototypisch in Form eines Klick-Dummys in Figma umgesetzt. Dabei wurde ein reduziertes, kartenbasiertes Interface mit klarem Fokus auf Sicherheit und einfache Bedienung entwickelt. Zentrale Funktionen wie Notfall-SOS, Live-Tracking, Routen-Check und Sturzerkennung wurden bewusst prominent und intuitiv integriert. Die Evaluation des Prototyps zeigte eine sehr positive Nutzererfahrung. Besonders gelobt wurden die Attraktivität, Steuerbarkeit und pragmatische Qualität. Die durchgeführten Benutzertests zeigten jedoch auch Optimierungspotenziale bei Bedienelementen, kritischen Funktionen und dem Systemfeedback sowie bei zusätzlichen Funktionen, wie der Integration eines Onboardings oder Community-Features.

Die in der Evaluation durchgeführten Benutzertests lieferten ein sehr positives Feedback zur Attraktivität des Prototyps und dessen Steuerbarkeit. Dennoch gab es Optimierungspotenzial bei den Bedienelementen und dem Systemfeedback. Zudem wurde ein Onboarding-Prozess als Ergänzung angeregt. Abschließend wurde für die App-Einführung ein ganzheitliches Supportkonzept über vier Phasen entwickelt. Dieses kombiniert bedarfsorientierte Maßnahmen für Nutzende und deren Notfallkontakte. Ein kontinuierlicher Iterationszyklus sichert dabei langfristig die Qualität und den Erfolg der Anwendung.

Zukünftige Iterationen sollten sich neben der technischen Weiterentwicklung vor allem auf die Einbindung der Angehörigen sowie die finale Namensgebung fokussieren. Da die Perspektive der Angehörigen bisher fehlt, muss diese Zielgruppe nun gezielt adressiert werden. Sie spielt besonders für das Live-Tracking und Notfallbenachrichtigungen eine zentrale Rolle. Bei der Namenswahl zeigten sich in der Evaluation eine klare Präferenz für „MoveSafe“ und „MoveGuard“. Im nächsten Schritt sollte zudem geprüft werden, ob ein deutscher Name den Zugang für ältere Zielgruppen erleichtert.
