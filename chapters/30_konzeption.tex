\chapter{Konzeption}
Die in den vorangegangenen Kapiteln gewonnenen Erkenntnisse werden in diesem Kapitel in die Konzeption von Personas, Szenarien sowie einer Aufgabenanalyse-Matrix überführt. Der Fokus dieser Phase liegt bewusst auf der Entwicklung für die Nutzergruppe der Aktiven. Aufgrund der fehlenden Perspektive der Angehörigen, nicht zuletzt bedingt durch fehlerhafte Daten aus der Online-Umfrage, besteht Klärungsbedarf hinsichtlich der notwendigen Funktionalitäten und der konkreten Ausgestaltung des Prototyps für diese Zielgruppe. Die Berücksichtigung der Angehörigenperspektive ist somit nicht Teil der nachfolgenden Konzeption sowie Evaluation und Einführungsphase.

\section{Personas}

\subsection{Persona 1: Laura Stahl}
\begin{itemize}
\item \textbf{Alter:} 36 Jahre
    
\item \textbf{Beruf:} freie TV-Autorin beim Westdeutschen Rundfunk, hohe Reisetätigkeit für Dreharbeiten und unregelmäßige Arbeitszeiten

\item \textbf{Familien-Background:} Lebt mit ihrem Partner in Köln-Ehrenfeld (keine Kinder). Ihr Partner arbeitet in Mainz und ist daher nur am Wochenende zu Hause. Sie verbringt die Wochentage oft allein.

\item \textbf{Outdoor-Aktivitäten:} Geht spontan nach Feierabend oder in der Mittagspause allein spazieren (oft in städtischen Parks oder entlang des Rheins), um Stress abzubauen und den Kopf freizubekommen. Meistens ist das erst nach Einbruch der Dunkelheit der Fall.

\item \textbf{IT-Vorkenntnisse:} gute Kenntnisse, sie nutzt häufig dieselben Apps und digitalen Tools vor allem in ihrem Job

\item \textbf{Endgeräte:} iPhone, Apple Watch mit LTE, Bluetooth-Kopfhörer

\item \textbf{Häufig genutzte Anwendungen:} WhatsApp, Google Maps, Apple Health, DB Navigator, Spotify

\item \textbf{Persönliche Ziele:} Sie möchte sich bei ihren Spaziergängen in vertrauter Umgebung in Köln sowie auf Reisen jederzeit sicher fühlen, unabhängig von Dunkelheit oder Tageszeit. Sie sucht nach Wegen, um sich schnell bemerkbar zu machen oder vertrauenswürdige Kontakte bzw. im Notfall Hilfsdienste zu alarmieren, falls sie sich unwohl fühlt.

\item \textbf{Einstellungen:} Ihre größte Sorge sind Überfälle oder Belästigungen, besonders wenn sie allein in unbekannten Gebieten oder Brennpunktvierteln in Großstädten unterwegs ist. Sie legt großen Wert auf eine intuitive App-Bedienung, die über eine Notfall-SOS-Funktion, eine transparente Standortfreigabe und zuverlässige Warnhinweise verfügt.

\end{itemize}

\subsection{Persona 2: Rosa Leitner}
\label{sec:rosa}

\begin{itemize}
\item \textbf{Alter:} 25 Jahre

\item \textbf{Beruf:} Studentin der Umweltwissenschaften an der Uni Freiburg

\item \textbf{Familien-Background:} Lebt in einer WG am Stadtrand von Freiburg. Ihre Familie wohnt in  Hinterzarten. Sie nutzt häufig öffentliche Verkehrsmittel oder das Fahrrad.

\item \textbf{Outdoor-Aktivitäten:} Trailrunning am Wochenende im Schwarzwald, regelmäßige Wanderungen mit Kommilitoninnen oder allein in entlegenen Gebieten im Schwarzwald oder mehrtägige Touren in den Semesterferien in Skandinavien

\item \textbf{IT-Vorkenntnisse:} Gute Kenntnisse, ist technikaffin und nutzt digitale Tools zur Kartierung und Datenanalyse im Studium

\item \textbf{Endgeräte:} Android-Smartphone, Polar Multisportuhr, zusätzliches einfaches GPS-Gerät ohne Mobilfunk, Windows Laptop

\item Häufig genutzte Anwendungen: Outdooractive, adidas Running, Bergfex Wetter, Google Maps, Signal

\item \textbf{Persönliche Ziele:} Sie möchte die Natur unabhängig und frei genießen, dabei aber jederzeit sicher unterwegs sein. Besonders wichtig sind ihr Offline-Funktionen, zuverlässige GPS-Ortung sowie Notfallfunktionen ohne Mobilfunknetz. Sie möchte Risiken wie Stürze oder Verlaufen vermeiden.

\item \textbf{Einstellungen:} Sie fürchtet sich davor, in entlegenen Gegenden allein zu sein, ohne dass schnelle Hilfe erreichbar ist. Eng damit verbunden ist die Angst vor Stürzen oder Verletzungen auf unebenen Pfaden. Aufgrund dieser Bedenken ist ihr eine zuverlässige technische Ausstattung besonders wichtig. Sie verlässt sich auf Offline-Karten zur Navigation, eine Notfall-SOS-Funktion und eine automatische Bewegungserkennung. Ihre Standortfreigabe nutzt sie nur sehr bewusst und situativ, meistens gegenüber ihrer Familie oder ihrer WG.


\end{itemize}

\subsection{Persona 3: Markus Schneider}
\begin{itemize}
\item Alter: 48 Jahre

\item \textbf{Beruf:} Projektmanager im Bereich Windkraft/erneuerbare Energien

\item \textbf{Familien-Background:} Verheiratet, zwei Kinder, lebt in Hamburg-Harburg

\item \textbf{Outdoor-Aktivitäten:} Täglicher Radpendler (ca. 20 km pro Strecke), Rennradtouren im Großraum Hamburg am Wochenende, Wanderurlaub mit der Familie

\item \textbf{IT-Vorkenntnisse:} Sehr gute Kenntnisse, da er technische Projektsoftware, Geoinformationen und digitale Planungstools in seinem Job nutzt.

\item \textbf{Endgeräte:} Android-Smartphone, Smartwatch und GPS-Radcomputer (beides Garmin), Windows Laptop

Häufig genutzte Anwendungen: WhatsApp, Komoot, Google Maps, Wetter.com, Strava und Garmin Connect
\item \textbf{Persönliche Ziele:} Er möchte unabhängig von Wetter und Tageszeit sicher mit dem Rad unterwegs sein. Selbst dann, wenn er allein und im Dunkeln fährt. Er nutzt Apps gerne, um Risiken frühzeitig zu erkennen und Touren vorausschauend zu planen.

\item \textbf{Einstellungen:} Stürze auf dem Rad sind seine größte Sorge. Er hat außerdem eine hohe Sensibilität für Sicherheit und Datenschutz. Ebenso legt er Wert auf Offline-Karten, zuverlässige Navigationsdaten und proaktive Gefahrenhinweise. Er erwartet eine intuitive und funktionsstarke App.


\end{itemize}

\section{Szenarien}

\subsection{IST-Szenario Rosa Leitner}
Rosa ist begeisterte Trailrunnerin und lebt in einer WG im Schwarzwald. An einem Herbstmorgen plant sie eine Trainingsrunde auf einer abgelegenen Waldstrecke. Beim Frühstück bespricht sie mit ihren Mitbewohnerinnen kurz die Wetterlage, da es in der Nacht geregnet hat, mögliche Routen und erinnert sich daran, dass der Mobilfunkempfang auf den Strecken oft schlecht ist. Sicherheitshalber lädt sie die Offline-Karte aus ihrer Komoot-App herunter und speichert die geplante Strecke auf ihrem Smartphone. Zur Absicherung schreibt sie noch eine Nachricht an eine Freundin mit der gewählten Strecke und der voraussichtlichen Dauer. Nach einem kurzen Warm-up startet Rosa den Lauf. Nach wenigen Kilometern bemerkt sie, dass ihr Handy wieder mal kaum Empfang hat. Die Strecke ist matschig, rutschig und teilweise schwer zu erkennen. Rosa konzentriert sich stark auf den Weg und fühlt sich angespannt, weil sie alleine unterwegs ist und im Notfall niemanden erreichen kann. Sie versucht, ihr Lauftempo beizubehalten, obwohl sie sich unwohl fühlt. Nach einem steilen Abhang rutscht Rosa plötzlich auf nassen Wurzeln aus und stürzt. Rosa realisiert, dass sie weder den Notruf absetzen noch jemanden anrufen kann, da kein Signal vorhanden ist. Verunsichert und unter Schmerzen steht sie auf, bricht ihren Trailrun ab und humpelt nach Hause. Ihr wird klar, wie lückenhaft ihre Absicherung unterwegs ist. Apps unterstützen sie zwar, aber wirklich proaktive Sicherheitsfunktionen fehlen und das besonders dort, wo sie sie am dringendsten braucht.

\subsection{IST-Szenario für Persona Markus Schneider}
Es ist 5:30 Uhr an einem kalten Wintermorgen. Markus steht auf, während es draußen noch stockdunkel ist. Seine Familie schläft noch. Bevor er losfährt, prüft er wie jeden Tag in mehreren Apps die Verkehrslage und das Wetter für seinen 20-Kilometer-Radweg von Hamburg-Harburg in die Innenstadt. Nachdem er sich seine reflektierende Kleidung angezogen und seinen Garmin-Radcomputer gestartet hat, beginnt seine Fahrt durch wechselnde Lichtverhältnisse. Besonders die dunklen Abschnitte hasst er. Glatte Stellen, Äste oder blockierte Wege sind kaum zu erkennen und werden von keiner App vorhergesagt. Im Büro angekommen synchronisiert Markus seine Fahrt wie gewohnt mit Strava und Garmin Connect. Anschließend sendet er seiner Frau eine kurze WhatsApp-Nachricht, dass er gut angekommen ist und es draußen sehr glatt ist. Sie möge bei ihrer morgendlichen Laufrunde doch bitte besonders gut aufpassen. 
Insgesamt funktioniert sein täglicher Ablauf zwar verlässlich, ist jedoch fragmentiert. Die Informationen sind verstreut, Hinweise häufig unvollständig und eine wirklich proaktive Sicherheitsunterstützung fehlt völlig. Keine App warnt ihn beispielsweise gezielt vor relevanten Gefahren wie Glätte, schlecht beleuchteten Bereichen oder spontanen Hindernissen. Gerade jetzt an dunklen, winterlichen Tagen wünscht sich Markus eine App, die ihm frühzeitig Orientierung und Sicherheit bietet, anstatt dass er sich alle relevanten Informationen selbst zusammensuchen muss.

\section{Aufgabenanalyse-Matrix}
