\chapter{Evaluation}
In diesem Kapitel wird die Evaluation des Prototyps erläutert. Die gewonnenen Erkenntnisse aus den Benutzertests dienen der künftigen Optimierung der Sicherheits-App. Diese ist jedoch nicht mehr Teil dieser Projektarbeit.

\section{Benutzertests}
\subsection{Methodik und Durchführung}
Im Rahmen dieser Arbeit wurden Benutzertests unter Verwendung der Langversion des standardisierten „User ExperienceQuestionnaire“ (UEQ) mit 26 Items durchgeführt \autocite{ueqonline}.
Die Auswertung der Fragebögen erfolgte mit dem zugehörigen UEQ Data Analysis Tool \autocite{ueqonline}. Die integrierte Benchmark-Funktion ermöglichte dabei einen direkten Vergleich der gemessenen User Experience der Sicherheits-App mit den Ergebnissen etablierter Produkte aus der Benchmark-Datenbank.

Zur Durchführung der Benutzertests absolvierten die Testpersonen unter Beobachtung fünf realistische Nutzungsszenarien (siehe \autoref{app:evaluation}, \autoref{fig:leitfaden_test_1}). Die Bearbeitung dieser Testaufgaben erfolgte sequenziell unter Anwendung der Methode des „Lauten Denkens“. In einer abschließenden Feedbackrunde wurden die Teilnehmenden zu ihren positiven Eindrücken, Kritikpunkten und Verbesserungsvorschlägen sowie zur künftigen Nutzungswahrscheinlichkeit befragt. Zudem sollten sie aus einer Liste von zehn Namensvorschlägen einen Favoriten für die Anwendung auswählen \autoref{app:evaluation}, \autoref{fig:leitfaden_test_2}). Bislang hatte diese noch den Arbeitstitel „Sicherheits-App für Outdooraktivitäten“.

Die Benutzertests begannen mit einer Erläuterung des Ablaufs und der Klärung offener Fragen. Die Bearbeitungszeit war unbegrenzt, beanspruchte jedoch im Durchschnitt 20 bis 30 Minuten. Nach Abschluss der Aufgaben füllten die Probanden den UEQ-Fragebogen (siehe \autoref{app:evaluation}, \autoref{fig:ueq_fragenbogen}) aus. Die Durchführung erfolgte ohne Aufzeichnung sowohl remote als auch in Präsenz. Über einen Figma-Link konnten die Testpersonen den Prototyp auf ihrem privaten Rechner oder einem bereitgestellten Gerät aufrufen. Auf Tests direkt am Smartphone wurde verzichtet, um Darstellungsfehler durch verschiedene Displaygrößen zu vermeiden, da der Figma-Prototyp lediglich auf das iPhone 16 ausgelegt ist.


Die Auswahl der Testpersonen erfolgte mit dem Ziel, die drei definierten Personas (siehe \autoref{app:personas}) bestmöglich abzubilden. Bei der Rekrutierung wurde neben der Teilnahme an der vorangegangenen Online-Umfrage auf eine ausgewogene Verteilung von Geschlecht und Alter geachtet. Potenzielle Probanden wurden direkt kontaktiert und bei Interesse zu einem individuellen (Online-)Benutzertest eingeladen.


\subsection{Auswertung}



\section{Optimierungsmaßnahmen}