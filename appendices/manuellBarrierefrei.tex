\chapter{Manuelle Überprüfung Barrierefrei-Seite}

\begin{xltabular}{\linewidth}{p{2.5cm} p{2.5cm} p{4.5cm} X}
    % --- Definition Kopfzeile (Erste Seite) ---
    \caption{Manuelle Testergebnisse Föjr: Barrierefrei} \label{tab:testergebnisse-barrierefrei} \\
    \toprule
    \textbf{Prüfschritt} & \textbf{Bewertung} & \textbf{Begründung / \newline Verbesserung} & \textbf{Weitere Notizen} \\
    \midrule
    \endfirsthead

    % % --- Definition Kopfzeile (Folgeseiten) ---
    % \caption[]{Testergebnisse Startseite (Fortsetzung)} \\
    % \toprule
    % \textbf{Prüfschritt} & \textbf{Bewertung} & \textbf{Begründung / \newline Verbesserung} & \textbf{Weitere Notizen} \\
    % \midrule
    % \endhead
    
    % --- Fußzeilen ---
    \midrule
    \multicolumn{4}{r}{\textit{Weiter auf der nächsten Seite...}} \\
    \endfoot

    \bottomrule
    \endlastfoot

    % --- Inhalt (Sortiert) ---
    
    % 1. Eintrag (war vorher unten)
    9.1.1.1a & 
    Teilweise erfüllt & 
    Fehlendes alt-Attribut für ein Bild
 & 
    Bild "`Strandkörbe für Rollifahrer"' \\ 
    \midrule


    % 2. Eintrag (war vorher oben)
    9.1.1.1b & 
    Teilweise erfüllt & 
    alt-Texte könnten insgesamt etwas beschreibender sein, z. B. statt alt="'Friesentorte"'
alt="'Nahaufnahme einer Friesentorte mit Kaffee auf Gartentisch eines Cafés"'
oder statt alt="`Rollstuhl, Koffer oder Kinderwagen die Fähre ist barrierefrei erreichbar"'
alt="`Innenansicht Seiteneinstieg mit flacher, barrierearmer Rampe am Fähranleger in Dagebüll"'

 & Bild "`Barrierefreie Gastronomie"' und "`Barrierefreie Anreise"'
    \\ 
     \midrule

    % 2. Eintrag (war vorher oben)
    9.1.1.1c & 
    Vollständig erfüllt & 
      & 
    \\ \midrule
    
    9.1.1.1d & 
    Nicht anwendbar & 
      & 
    \\ \midrule
    
    9.1.3.1a & 
    Teilweise erfüllt & 
    Es gibt eine leere H3-Überschrift auf die eine weitere H3-Überschrift mit Text folgt. Die leere Überschrift muss entfernt werden. & 
    Barrierefreies AQUAFÖHR (zwischen Sanitärräume und Promenadenzugang)
    \\ \midrule

    9.1.3.1b & 
    Vollständig erfüllt & 
      & 
    \\ \midrule

    9.1.3.1c & 
    Nicht anwendbar & 
     & 
    \\ \midrule

     9.1.3.1d & 
    Teilweise erfüllt & 
    Ein main-Element ist innerhalb eines weiteren main-Elements verschachtelt. Pro Seite darf allerdings nur ein main-Element existieren.
Das innere main-Element muss entfernt oder durch section, div oder article ersetzt werden.
Es gibt außerdem leere p-Elemente, um vertikalen Abstand zu erzeugen.Gewünschter Abstand sollt besser über CSS gesteuert werden

 & 
    \\ \midrule

    9.1.3.1e & 
    Nicht anwendbar & 
    
 & 
    \\ \midrule

    9.1.3.1f & 
    Nicht anwendbar & 
    
 & 
    \\ \midrule

    9.1.3.1g & 
    Vollständig erfüllt & 
    
 & 
    \\ \midrule

    9.1.3.1h & 
    Vollständig erfüllt & 
 & 
    \\ \midrule

    9.1.3.2 & 
    Vollständig erfüllt & 
     & 
    \\ \midrule

    9.1.3.3 & 
    Vollständig erfüllt & 
     & 
    \\ \midrule

    9.1.3.4 & 
    Vollständig erfüllt & 
     & 
    \\ \midrule

    9.1.4.1 & 
    Vollständig erfüllt & 
     & 
    \\ \midrule

    9.1.4.2 & 
    Nicht anwendbar & 
     & 
    \\ \midrule

    9.1.4.3 & 
    Teilweise erfüllt & 
    Zu geringer Kontrast, teilweise sogar nur 1,96:1, betrifft vor allem Text, der direkt auf Bildern platziert ist, insbesondere Copyright-Angaben, sowie spezifische Farbkombinationen im Layout, z. B. die hellgrüne Textfarbe auf beigefarbenem Hintergrund & 
    \\ \midrule

    9.1.4.4 & 
    Vollständig erfüllt & 
     & 
    \\ \midrule

    9.1.4.5 & 
    Vollständig erfüllt & 
     & 
    \\ \midrule

    9.1.4.10 & 
    Vollständig erfüllt & 
     & 
    \\ \midrule

    9.1.4.11 & 
    Vollständig erfüllt & 
     & 
    \\ \midrule

    9.1.4.12 & 
    Vollständig erfüllt & 
     & 
    \\ \midrule

    9.1.4.13 & 
    Nicht erfüllt & 
    Tooltip bei "`merken"' verschwindet nicht bei "`esc"' Tastendruck & 
    \\ \midrule

    9.2.1.1 & 
    Vollständig erfüllt & 
     & 
    \\ \midrule

     9.2.1.2 & 
    Vollständig erfüllt & 
     & 
    \\ \midrule
    
    9.2.1.4 & 
    Vollständig erfüllt & 
     & 
    \\ \midrule

     9.2.3.1 & 
    Vollständig erfüllt & 
     & 
    \\ \midrule

     9.2.4.1 & 
    Teilweise erfüllt & 
    Ein main-Element ist innerhalb eines weiteren main-Elements verschachtelt. Pro Seite darf allerdings nur ein main-Element existieren.
Das innere main-Element muss entfernt oder durch section, div oder article ersetzt werden.
 & 
    \\ \midrule

    9.2.4.2 & 
    Vollständig erfüllt & 
     & 
    \\ \midrule

    9.2.4.3 & 
    Vollständig erfüllt & 
     & 
    \\ \midrule

    9.2.4.4 & 
    Teilweise erfüllt & 
    Link-Text ist mit "`Mehr erfahren"' nicht aussagekräftig, besser wäre hier "`Infos zur barrierefreien Anreise"' & Artikel "`Barrierefreie Anreise
    \\ \midrule

    9.2.4.5 & 
    Nicht erfüllt & 
    Unterseite ist nur über das Suchfeld auffindbar und z. B. nicht auf der Seite "`Barrierefreie Anreise"' verlinkt & 
    \\ \midrule

    9.2.4.6 & 
    Vollständig erfüllt & 
     & 
    \\ \midrule

    9.2.4.7 & 
    Teilweise erfüllt & 
    Der Standard-Fokusindikator ist aufgrund der bereits vorhandenen leichten Rahmen (Kästen) der Buttons visuell nur schwer vom normalen Zustand des Elements zu unterscheide. Besser wäre eine zusätzliche, kontrastreiche farbliche Hervorhebung (z. B. einen dicken, farbigen Rahmen oder eine Hintergrundfarbänderung)& 
    Alle Buttons mit leichtem Rahmen
    \\ \midrule

    9.3.1.1 & 
    Vollständig erfüllt & 
     & 
    \\ \midrule

    9.3.1.2 & 
    Nicht anwendbar & 
    & 
    \\ \midrule

    9.4.1.1 & 
    Teilweise erfüllt & 
    Es gibt verschachtelte main-Elemente, leere id-Attribute, doppelte id-Werte, ungültige Attributwerte (required) sowie unzulässige Elementverschachtelung (p innerhalb span). 

Zweites main-Element entfernen, gültige, eindeutige IDs sowie required="required" verwenden, p entweder direkt verwenden oder muss oder n ein div-Element einbetten. 
 & 
    \\ \midrule

    9.4.1.2 & 
    Vollständig erfüllt & 
     & 
    \\ \midrule

    9.4.1.3 & 
    Vollständig erfüllt & 
     & 
    \\
\end{xltabular}

