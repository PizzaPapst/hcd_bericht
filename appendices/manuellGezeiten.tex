\chapter{Manuelle Überprüfung Gezeiten-Seite}

\begin{xltabular}{\linewidth}{p{2.5cm} p{2.5cm} p{4.5cm} X}
    % --- Definition Kopfzeile (Erste Seite) ---
    \caption{Manuelle Testergebnisse Föhr zwischen Ebbe und Flut: Gezeitenkalender} \label{tab:testergebnisse-gezeiten} \\
    \toprule
    \textbf{Prüfschritt} & \textbf{Bewertung} & \textbf{Begründung / \newline Verbesserung} & \textbf{Weitere Notizen} \\
    \midrule
    \endfirsthead

    % % --- Definition Kopfzeile (Folgeseiten) ---
    % \caption[]{Testergebnisse Startseite (Fortsetzung)} \\
    % \toprule
    % \textbf{Prüfschritt} & \textbf{Bewertung} & \textbf{Begründung / \newline Verbesserung} & \textbf{Weitere Notizen} \\
    % \midrule
    % \endhead
    
    % --- Fußzeilen ---
    \midrule
    \multicolumn{4}{r}{\textit{Weiter auf der nächsten Seite...}} \\
    \endfoot

    \bottomrule
    \endlastfoot

    % --- Inhalt (Sortiert) ---
    
    % 1. Eintrag (war vorher unten)
    9.1.1.1a & 
    Teilweise erfüllt & 
    Links mit href="`javascript:void(0)"' für "`Artikel weiterlesen/weniger anzeigen"' haben aria-hidden="`true"' - diese sollten echte Buttons sein. Der Submit-Button der Suche hat nur title="`Suche abschicken"', aber keinen sichtbaren Text. Besser wäre ein aria-label. Der "`merken"'-Button hat keinen ausreichend beschreibenden Text für Screenreader.
 & 
    Bild "`Strandkörbe für Rollifahrer"' \\ 
    \midrule


    % 2. Eintrag (war vorher oben)
    9.1.1.1b & 
    Teilweise erfüllt & 
    Lighthouse meldet ein fehlendes alt-Attribut bei einem Bild.
 & Bild: "`FÖHRgreen"'
    \\ 
     \midrule

    % 2. Eintrag (war vorher oben)
    9.1.1.1c & 
    Vollständig erfüllt & 
      & 
    \\ \midrule
    
    9.1.1.1d & 
    Nicht anwendbar & 
      & 
    \\ \midrule
    
    9.1.3.1a & 
    Vollständig erfüllt & 
    Überschriften sind vorhanden und strukturiert (H1: "`Gezeiten"', H2/H3: Unterüberschriften). Die Hierarchie erscheint logisch. & 
    \\ \midrule

    9.1.3.1b & 
    Vollständig erfüllt & 
      & 
    \\ \midrule

    9.1.3.1c & 
    Nicht anwendbar & 
     & 
    \\ \midrule

     9.1.3.1d & 
    Vollständig erfüllt & 
    Der Inhalt ist in logische Abschnitte gegliedert (Gezeiteninfo, Sicherheitshinweise, Wassertemperatur, Wetter)

 & Gute visuelle und semantische Gliederung erkennbar.
    \\ \midrule

    9.1.3.1e & 
    Nicht anwendbar & 
    
 & 
    \\ \midrule

    9.1.3.1f & 
    Nicht anwendbar & 
    
 & 
    \\ \midrule

    9.1.3.1g & 
    Vollständig erfüllt & 
    
 & 
    \\ \midrule

    9.1.3.1h & 
    Nicht erfüllt &  
    Formelemente benötigen ein Label. Das Label fehlt beim Datumspicker. Datepicker unter "`Gezeitenkurve Wyk auf Föhr"' &
    Mit dem Jahreswechsel hat sich die Webseite geändert. Es gibt keine Gezeitenkurve mehr, sondern stattdessen ein downloadbares PDF. Die Analyse fand davor statt.
    \\ \midrule

    9.1.3.2 & 
    Vollständig erfüllt & 
    Die Inhaltsreihenfolge ist logisch: Navigation, Hauptinhalt, Zusatzinformationen, Footer. & 
    Die Lesereihenfolge entspricht der visuellen Reihenfolge.
    \\ \midrule

    9.1.3.3 & 
    Vollständig erfüllt & 
    Keine rein visuellen Anweisungen werden auf der Webseite verwendet. & 
    Inhalte sind auch ohne Farb- oder Positionsinformationen verständlich.
    \\ \midrule

    9.1.3.4 & 
    Vollständig erfüllt & 
     & 
    \\ \midrule

    9.1.4.1 & 
    Vollständig erfüllt & 
     & 
    \\ \midrule

    9.1.4.2 & 
    Nicht anwendbar & 
     & 
    \\ \midrule

    9.1.4.3 & 
    Teilweise erfüllt & 
    Im Header hat die weiße Schrift auf dem Video-Hintergrund einen Kontrast von 1.1 und damit gering. &
    Auf manchen Bildern unter "`Das könnte Sie auch interessieren…"' ist die Schrift des Copyrights Teil des Fotos und nicht als Schrift zu inspektieren. Man kann annehmen, dass hier der Kontrast nicht ausreichend wäre. 
    \\ \midrule

    9.1.4.4 & 
    Vollständig erfüllt & 
     & 
    \\ \midrule

    9.1.4.5 & 
    Vollständig erfüllt & 
     & 
    \\ \midrule

    9.1.4.10 & 
    Vollständig erfüllt & 
     & 
    \\ \midrule

    9.1.4.11 & 
    Vollständig erfüllt & 
     & 
    \\ \midrule

    9.1.4.12 & 
    Vollständig erfüllt & 
     & 
    \\ \midrule

    9.1.4.13 & 
    Vollständig erfüllt & 
     & 
    \\ \midrule

    9.2.1.1 & 
    Vollständig erfüllt & 
     & 
    \\ \midrule

     9.2.1.2 & 
    Vollständig erfüllt & 
     & 
    \\ \midrule
    
    9.2.1.4 & 
    Vollständig erfüllt & 
    Keine benutzerdefinierten Tastatur-Kurzbefehle erkennbar. & 
    \\ \midrule

     9.2.3.1 & 
    Vollständig erfüllt & 
    Keine automatisch animierten oder flackernden Inhalte erkennbar.     & 
    \\ \midrule

     9.2.4.1 & 
    Nicht anwendbar & 
    Es sind keine Sprunglinks vorhanden.& 
    \\ \midrule

    9.2.4.2 & 
    Vollständig erfüllt & 
    Seitentitel ist aussagekräftig: "Föhr zwischen Ebbe und Flut: Gezeitenkalender" & 
    \\ \midrule

    9.2.4.3 & 
    Vollständig erfüllt & 
     & 
    \\ \midrule

    9.2.4.4 & 
    Teilweise erfüllt & 
    Unter "`Das könnte Sie auch interessieren…"', haben die verlinkten Seiten alle den Text "`will ich sehen"'. Hier könnte etwas aussagekräftiges gewählt werden. Zum Beispiel "`Mehr erfahren"' für den Link zum Schlafstrandkorb. &
    \\ \midrule

    9.2.4.5 & 
    Vollständig erfüllt & 
    Mehrere Navigationswege vorhanden: Hauptmenü, Footer-Links, Breadcrumbs (zurück zur Startseite). & 
    \\ \midrule

    9.2.4.6 & 
    Vollständig erfüllt & 
     & 
    \\ \midrule

    9.2.4.7 & 
    Vollständig erfüllt & 
    Standard-Browser-Fokusring ist meistens sichtbar.& 
    \\ \midrule

    9.3.1.1 & 
    Vollständig erfüllt & 
    Die Seite ist auf Deutsch und hat lang="`de"' im HTML-Tag. & 
    \\ \midrule

    9.3.1.2 & 
    Nicht anwendbar & 
    &  Im Menü gibt es den Link "`På dansk"', der selbsterklärend ist und zu einer Seite mit Informationen zu Föhr auf Dänisch weiterleitet.
    \\ \midrule

    9.4.1.1 & 
    Teilweise erfüllt & 
    Blockquote wird falsch verwendet. & 
    \\ \midrule

    9.4.1.2 & 
    Vollständig erfüllt & 
     & 
    \\ \midrule

    9.4.1.3 & 
    Nicht Anwendbar für Newsletter, da die Seite nach Anmeldung neu geladen wird & 
     & 
    \\
\end{xltabular}

