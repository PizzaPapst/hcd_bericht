\chapter{WAVE Testergebnisse autoamtische Analyse}
\label{app:autoWAVE}
\begin{xltabular}{\linewidth}{p{2.5cm} p{2.5cm} p{4.5cm} X}
    % --- Kopfzeile (Erste Seite) ---
    \caption{WAVE Testergebnisse Startseite} \label{tab:testergebnisse-startseite-wave} \\
    \toprule
    \textbf{Prüfschritt} & \textbf{Bewertung} & \textbf{Begründung / \newline Verbesserung} & \textbf{Weitere Notizen} \\
    \midrule
    \endfirsthead

    % % --- Kopfzeile (Folgeseiten) ---
    % \caption[]{Testergebnisse Föhr: Barrierefrei} \\
    % \toprule
    % \textbf{Prüfschritt} & \textbf{Bewertung} & \textbf{Begründung / \newline Verbesserung} & \textbf{Weitere Notizen} \\
    % \midrule
    % \endhead

    % --- Fußzeilen ---
    \midrule
    \multicolumn{4}{r}{\textit{Weiter auf der nächsten Seite...}} \\
    \endfoot

    \bottomrule
    \endlastfoot

    % --- Inhalt (Sortiert) ---

    % 1. Eintrag (9.1.1 comes first)
    9.1.1.1a & 
    Teilweise erfüllt & 
    Fehlende alt-Attribute für vier Bilder & 
    Hero Image, Bild für "`Mach mal Föhr"', "`Veranstaltungen auf Föhr"' und "`Ader-Schiffe"' \\ 
    \midrule

    % 2. Eintrag (9.1.4.2 comes before 9.1.4.3)
    9.1.4.3 & 
    Teilweise erfüllt & 
    Zu geringer Kontrast, teilweise sogar nur 1,96:1, betrifft vor allem Text, der direkt auf Bildern platziert ist, insbesondere Copyright-Angaben, sowie spezifische Farbkombinationen im Layout, Z. B. die hellgrüne Textfarbe auf beigefarbenem Hintergrundenthalten. & 
    \\ 
    \midrule

    % 3. Eintrag
    9.2.4.4 & 
    Teilweise erfüllt & 
    Der Schließen-Link im Suchfeld zeigt zwar ein Icon an, hat aber keinen Namen für Screenreader (das Icon ist per aria-hidden="true" versteckt) und das title-Attribut ist nicht ausreichend \rightarrow a-Element mit aria-label="`Suchformular schließen"', um Screenreadern einen eindeutigen Linkzweck zu vermitteln. Außerdem sind die Vergrößerungs-Links in der Mediengalerie leer & 
    Suchfeld im Header Mediengalerie "`Erlebe deinen Nordseeurlaub auf der Insel Föhr"'
 \\ 
 \midrule

 % 3. Eintrag
    9.3.3.2 & 
    Teilweise erfüllt & 
    Fehlen im in zwei von drei Formularen und dort jeweils für alle Eingabefelder& 
    Quicksearch Unterkunft (Datepicker und Personenanzahl), die zweimal auf der Seite eingebunden ist
 \\ 

\end{xltabular}

\begin{xltabular}{\linewidth}{p{2.5cm} p{2.5cm} p{4.5cm} X}
    % --- Kopfzeile (Erste Seite) ---
    \caption{WAVE Testergebnisse Föhr: Barrierefrei} \label{tab:testergebnisse-barrierefrei-wave} \\
    \toprule
    \textbf{Prüfschritt} & \textbf{Bewertung} & \textbf{Begründung / \newline Verbesserung} & \textbf{Weitere Notizen} \\
    \midrule
    \endfirsthead

    % % --- Kopfzeile (Folgeseiten) ---
    % \caption[]{Testergebnisse Föhr: Barrierefrei} \\
    % \toprule
    % \textbf{Prüfschritt} & \textbf{Bewertung} & \textbf{Begründung / \newline Verbesserung} & \textbf{Weitere Notizen} \\
    % \midrule
    % \endhead

    % --- Fußzeilen ---
    \midrule
    \multicolumn{4}{r}{\textit{Weiter auf der nächsten Seite...}} \\
    \endfoot

    \bottomrule
    \endlastfoot

    % --- Inhalt (Sortiert) ---

    % 1. Eintrag (9.1.1 comes first)
    9.1.1.1a & 
    Teilweise erfüllt & 
    Fehlende alt-Attribute für ein Bild & 
    Bild "`Strandkörbe für Rollifahrer"' \\ 
    \midrule

    % 3. Eintrag
    9.2.4.6 & 
    Teilweise erfüllt & 
    Inkonsequente Überschriftenhierarchie, es wird im Artikel ein leeres h3 class="`csc-linkToTop"'>\&nbsp;</h3>-Tag als Abstandshalter verwendet und das p-Element zusätzlich zur rein visuellen Abstandserzeugung missbraucht  & 
    Im Artikel "Barrierefreies "`AQUAFÖHR"' zwischen "`Fitnessstudio"' und "`Sanitärräume"'\\ 

\end{xltabular}


\begin{xltabular}{\linewidth}{p{2.5cm} p{2.5cm} p{4.5cm} X}
    % --- Kopfzeile (Erste Seite) ---
    \caption{WAVE Testergebnisse Föhr zwischen Ebbe und Flut: Gezeitenkalender} \label{tab:testergebnisse-gezeiten-wave} \\
    \toprule
    \textbf{Prüfschritt} & \textbf{Bewertung} & \textbf{Begründung / \newline Verbesserung} & \textbf{Weitere Notizen} \\
    \midrule
    \endfirsthead

    % % --- Kopfzeile (Folgeseiten) ---
    % \caption[]{Testergebnisse Föhr: Barrierefrei} \\
    % \toprule
    % \textbf{Prüfschritt} & \textbf{Bewertung} & \textbf{Begründung / \newline Verbesserung} & \textbf{Weitere Notizen} \\
    % \midrule
    % \endhead

    % --- Fußzeilen ---
    \midrule
    \multicolumn{4}{r}{\textit{Weiter auf der nächsten Seite...}} \\
    \endfoot

    \bottomrule
    \endlastfoot

    % --- Inhalt (Sortiert) ---

    % 1. Eintrag (9.1.1 comes first)
    9.1.1.1a & 
    Teilweise erfüllt & 
    Fehlende alt-Attribute für Bilder und Videos & 
    YouTube-Video im Hero-Bereich und Bild "`FÖHRgreen"' unter "`Das könnte Sie auch interessieren…"' \\ 
    \midrule

    % 2. Eintrag (9.1.4.2 comes before 9.1.4.3)
    9.2.4.6 & 
    Teilweise erfüllt & 
    Lückenhafte Überschriftenhierarchie, Artikel beginnen mit einem p-Element und haben keine H2-Überschrift & 
    Über den einzelnen Artikeln "`Beachten Sie im Watt unbedingt folgende Hinweise:"' und  "`Das könnte Sie auch interessieren…"' \\ 
    \midrule

 % 3. Eintrag
    9.3.3.2 & 
    Teilweise erfüllt & 
    Fehlt im Formular-Eingabefeld bzw. ist leer & 
    Datepicker Gezeitenkurve
 \\ 

\end{xltabular}